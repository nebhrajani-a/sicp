% Created 2021-06-07 Mon 20:16
% Intended LaTeX compiler: xelatex
\documentclass[9pt]{report}
\usepackage{graphicx}
\usepackage{grffile}
\usepackage{longtable}
\usepackage{wrapfig}
\usepackage{rotating}
\usepackage[normalem]{ulem}
\usepackage{amsmath}
\usepackage{textcomp}
\usepackage{amssymb}
\usepackage{capt-of}
\usepackage[hidelinks]{hyperref}
\usepackage[a4paper, total={6in,9in}]{geometry}
         \usepackage{booktabs}
\usepackage{minted}
         \usepackage{tikz-cd}
\usepackage{parskip}
\usepackage[type={CC}, modifier={by-nc-sa}, version={4.0},]{doclicense}
\usepackage{forest}
\usepackage{sectsty}
\sectionfont{\fontsize{12}{15}\selectfont}
\subsectionfont{\fontsize{11}{11}\selectfont}
\setlength\parindent{0pt}
\usepackage{parskip}
\usepackage{pifont}
\makeatletter
\def\@makechapterhead#1{%
{\parindent \z@ \raggedright \normalfont
\ifnum \c@secnumdepth >\m@ne
\LARGE\bfseries \thechapter~
\fi
\interlinepenalty\@M
\LARGE \bfseries #1\par\nobreak
\vskip 10\p@
}}
\def\@makeschapterhead#1{%
{\parindent \z@ \raggedright
\normalfont
\interlinepenalty\@M
\Huge \bfseries  #1\par\nobreak
\vskip 10\p@
}}
\makeatother
\author{Nebhrajani A.V.}
\date{\today}
\title{MIT 6.001 1986 Video Notes}
\hypersetup{
 pdfauthor={Nebhrajani A.V.},
 pdftitle={MIT 6.001 1986 Video Notes},
 pdfkeywords={},
 pdfsubject={},
 pdfcreator={Emacs 25.2.2 (Org mode 9.3.6)},
 pdflang={English}}
\begin{document}

\maketitle
\tableofcontents


\chapter{Introduction}
\label{sec:orga548684}

\section{About}
\label{sec:org14146d8}
These are my notes of the twenty SICP lectures of June 1986,
produced by Hewlett-Packard Television. These videos are available
under a Creative Commons license. These videos should be included in
the directory \texttt{../videos}.

These notes aim to be concise and as example-heavy as possible. The
language used and referred to as ``Lisp'' is MIT-Scheme. These notes,
however, use the SICP language provided by Racket, a modern Scheme
dialect. This is because Racket's integration with Emacs and
\texttt{org-mode} is orders of magnitude better than MIT-Scheme's. In
general, all ``Lisp'' code looks exactly the same as in SICP, with the
exception of having to prefix some numbers with \texttt{\#i} to ensure
Racket treats them as imprecise.

\section{License}
\label{sec:org3b31a9b}
\doclicenseThis

\chapter{Lecture 1A: Overview and Introduction to Lisp}
\label{sec:orge0cc9f8}

Computer science isn't really a science, and it isn't really about
computers. Computer science is the study of how-to or imperative
knowledge (as opposed to declarative knowledge). To illustrate the
difference, consider:

$$y = \sqrt{x} \mathrm{~such~that~} y^2=x, y \geq 0$$

This is declarative, in that we could recognize if \(y\) is the square
root of \(x\) given \(x\) and \(y\), but we're no closer to knowing how to
\emph{find} \(y\) if we are given \(x\). Imperative knowledge would look
like:

To find the square root \(y\) of \(x\):
\begin{itemize}
\item Make a guess \(g\).
\item If \(g^2\) is close enough to \(x\), \(y=g\).
\item Otherwise, make a new guess equal to the average of \(g\) and \(x/g\).
\end{itemize}

This method will eventually come up with a \(g\) close enough to the
actual square root \(y\) of \(x\).

Computer science focuses on this kind of imperative knowledge, and,
specifically, how to communicate that knowledge to a computer.

\section{Managing Complexity: Key Ideas of 6.001}
\label{sec:orge5a8b57}
Computer science is also about managing complexity, in that large
programs that you can't hold in your head should still be manageable
and easy to work with. We explore this theme in 6.001 by learning
three key ideas:

\begin{itemize}
\item Black-box abstractions
\item Conventional interfaces
\item Metalinguistic abstraction.
\end{itemize}


\section{Let's Learn Lisp}
\label{sec:org72ea28d}
When learning a new language, always ask about its:
\begin{itemize}
\item Primitive elements,
\item Means of combination, and
\item Means of abstraction.
\end{itemize}

\subsection{Primitive Elements}
\label{sec:org8f91593}
These are numbers like 3, 17.4, or 5. Other primitives are
discussed later in the course.

\begin{minted}[]{racket}
4
17.4
5
\end{minted}

\begin{verbatim}
4
17.4
5
\end{verbatim}

\subsection{Means of Combination}
\label{sec:orgdfe2066}
Lisp's numerical primitives can be combined with ``operations'' such
as addition, written in prefix notation.

\begin{minted}[]{racket}
(+ 3 17.4 5)
\end{minted}

\begin{verbatim}
25.4
\end{verbatim}


Other basic operations are provided by Lisp, such as
multiplication and division. Of course, combinations can be
combined recursively:

\begin{minted}[]{racket}
(+ 3 (* 5 6) 8 2)
\end{minted}

\begin{verbatim}
43
\end{verbatim}


This should show you the tree structure inherent in all of Lisp:
\begin{center}
\begin{forest}
[+
[* [5] [6]] [8] [2]]
\end{forest}
\end{center}

In Lisp, () is the application of an operation or function in
prefix notation.

\subsection{Means of Abstraction}
\label{sec:org68ae085}

Abstraction can simply be done by naming things. Giving
complicated things a name prevents us from having to understand
how the thing the name refers to \emph{works}, and instead lets us
``abstractly'' use the name for our purposes.

\begin{minted}[]{racket}
(define a (* 5 5))
a
(* a a)
(define b (+ a (* 5 a)))
b
(+ a (/ b 5))
\end{minted}

\begin{verbatim}
25
625
150
55
\end{verbatim}


Now, it's often more useful to abstract away imperative how-to
knowledge. Consider:

\begin{minted}[]{racket}
(define (square x)
  (* x x))
\end{minted}

\begin{minted}[]{racket}

(square 10)
\end{minted}

\begin{verbatim}
100
\end{verbatim}


This defines \texttt{square} as a function taking a single argument \texttt{x},
and returning \texttt{(* x x)}. Note that this way of writing a define is
actually ``syntactic sugar'' for:

\begin{minted}[]{racket}
(define square
  (lambda (x)
    (* x x)))

(square 25)
\end{minted}

\begin{verbatim}
625
\end{verbatim}


\texttt{lambda (x)} means ``make a procedure that takes argument \texttt{x}''. The
second argument to lambda is the actual procedure body. The
\texttt{define} names this anonymous procedure \texttt{square}.

Just like we can use combinations recursively, so we can
abstractions. Consider:

\begin{minted}[]{racket}
(define (average x y)
  (/ (+ x y) 2))
\end{minted}

\begin{minted}[]{racket}


(define (mean-square x y)
  (average (square x)
           (square y)))

(mean-square 2 3)
\end{minted}

\begin{verbatim}
13/2
\end{verbatim}


Note the indentation: since Lisp is parenthesis heavy, we use
indentation. Good editors like Emacs should do this automatically.

\section{Case Analysis in Lisp}
\label{sec:orge556cb5}

To represent functions like:
$$abs(x) = \begin{cases}
   -x & x<0\\
   0 & x = 0\\
   x & x > 0
   \end{cases}$$
Lisp needs some form of conditional execution. In Lisp, this
function would look like:

\begin{minted}[]{racket}
(define (abs x)
  (cond ((< x 0) (- x))
        ((= x 0) 0)
        ((> x 0) x)))
(abs -3)
(abs 0)
(abs 5)
\end{minted}

\begin{verbatim}
3
0
5
\end{verbatim}


\texttt{cond} takes any number of arguments. Each argument must be
structured as \texttt{(predicate) (consequent)}. If \texttt{predicate} is true,
we do the \texttt{consequent}. Otherwise, we don't. Lisp also provides a
way to write conditionals that only have two branches (an if-else):

\begin{minted}[]{racket}
(define (abs x)
  (if (< x 0)
      (- x)
      x))
\end{minted}

\begin{minted}[]{racket}

(abs -11)
(abs 0)
(abs 33)
\end{minted}

\begin{verbatim}
11
0
33
\end{verbatim}


\texttt{cond} and \texttt{if} are syntactical sugar for each other. The Lisp
implementation picks any one and defines the other in terms of it.

We now know most of Lisp. Lisp doesn't have \texttt{do...while} or \texttt{for},
since anything a loop can do can be done via recursion.

\section{Finding Square Roots}
\label{sec:org14c8f04}

Remember our square root finding algorithm?

To find the square root \(y\) of \(x\):
\begin{itemize}
\item Make a guess \(g\).
\item If \(g^2\) is close enough to \(x\), \(y=g\).
\item Otherwise, make a new guess equal to the average of \(g\) and
\(x/g\).
\end{itemize}

Or, in Lisp,

\begin{minted}[]{racket}
(define (try g x)
  (if (good-enough-p g x)
      g
      (try (improve g x) x)))
\end{minted}

This is a form of programming called ``wishful thinking'': we assume
\texttt{good-enough-p} (good enough predicate) and \texttt{improve} are already
implemented. Now that we can try a guess and improve it till it's
good enough, we can write a simple square root function:

\begin{minted}[]{racket}
(define (sqrt x)
  (try 1 x))
\end{minted}

This function simply starts the guess at 1, then improves it. Let's
now write the functions we don't have:

\begin{minted}[]{racket}
(define (improve g x)
  (average g (/ x g)))
\end{minted}

\begin{minted}[]{racket}
(define (good-enough-p g x)
  (< (abs (- (square g) x))
     0.00001))
\end{minted}

This tests if \(g^2\) is within 0.0001 of \(x\). Putting it all
together, we can finally try to find square roots:

\begin{minted}[]{racket}







(sqrt #i2)
(sqrt #i3)
(sqrt #i4)
\end{minted}

\begin{verbatim}
1.4142156862745097
1.7320508100147274
2.0000000929222947
\end{verbatim}


\begin{quote}
\textbf{Note:} The \texttt{\#i4} is Racket's syntax for using imprecise
(decimals) instead of precise (fractions). Ignore it, and treat it
as the number \texttt{4}.
\end{quote}

See that \texttt{try} actually runs a loop, but does so recursively,
calling itself every time the \texttt{if} condition fails to improve the
guess. Also note that these functions can all be nested inside the
square root function to hide them from the outer scope, thus:

\begin{minted}[]{racket}
(define (sqrt x)
  (define (good-enough-p g)
    (define (square g)
      (* g g))
    (define (abs y)
      (if (< y 0)
          (- y)
          y))
    (< (abs (- (square g) x))
       0.0001))
  (define (improve g)
    (define (average y z)
      (/ (+ y z) 2))
    (average g (/ x g)))
  (define (try g)
    (if (good-enough-p g)
        g
        (try (improve g))))
  (try 1))

(sqrt #i2)
\end{minted}

\begin{verbatim}
1.4142156862745097
\end{verbatim}


This program should also show you a tree-like dependency of the
functions, with each function containing the definitions of the
functions it depends on. For someone using \texttt{sqrt}, all the functions
within it are hidden.

\begin{center}
\begin{forest}
[\texttt{sqrt}
[\texttt{try}
[\texttt{good-enough-p}
[\texttt{abs}] [\texttt{square}]]
[\texttt{improve}
[\texttt{average}]]
[\texttt{try}]]]
\end{forest}
\end{center}

This discipline of writing procedures is called lexical scoping.


\section{Inbuilt/Primitive Procedures Aren't Special}
\label{sec:org4f9a79a}

\begin{minted}[]{racket}

square
+
\end{minted}

\begin{verbatim}
#<procedure:square>
#<procedure:+>
\end{verbatim}

\chapter{Lecture 1B: Procedures and Processes, Substitution Model}
\label{sec:orgaaddfc4}

\section{Substitution Rule/Model}
\label{sec:org13faf98}
The substitution rule states that,

\begin{quote}
To evaluate an application:
\begin{itemize}
\item Evaluate the operator to get procedure.
\item Evaluate the operands to get arguments.
\item Apply procedure to arguments.
\begin{itemize}
\item Copy body of procedure.
\item Replace formal parameters with actual arguments.
\end{itemize}
\item Evaluate new body.
\end{itemize}
\end{quote}

Note that this isn't necessarily how the \emph{interpreter} evaluates a
Lisp application, but the substitution rule is a ``good enough''
model for our purposes.

\subsection{Kinds of Expressions in Lisp}
\label{sec:org3757546}
\begin{itemize}
\item Numbers (evaluate to ``themselves'')
\item Symbols (represent some procedure)
\item Combinations
\item \(\lambda\)-expressions (used to build procedures)
\item Definitions (used to name symbols)
\item Conditionals
\end{itemize}

We will focus our use of the substitution rule on the first three.
The last three are called ``special forms'', and we'll worry about
them later.

\subsection{Example}
\label{sec:org4a3b0ba}

Consider:

\begin{minted}[]{racket}

(define (sum-of-squares x y)
  (+ (square x) (square y)))

(sum-of-squares 3 4)
\end{minted}

\begin{verbatim}
25
\end{verbatim}


Let's try to apply the substitution rule to our application,

\begin{minted}[]{racket}
(sum-of-squares 3 4)
(+ (square 3) (square 4))
(+ (square 3) (* 4 4))
(+ (square 3) 16)
(+ (* 3 3) 16)
(+ 9 16)
25
\end{minted}

\section{Peano Arithmetic}
\label{sec:org1ca7c7d}

\subsection{Simple Peano Addition}
\label{sec:org05a1e5b}
Peano arithmetic defines addition as:

\begin{minted}[]{racket}
(define (pa+ x y)
  (if (= x 0)
      y
      (pa+ (dec x) (inc y))))
\end{minted}

\begin{minted}[]{racket}

(pa+ 3 4)
\end{minted}

\begin{verbatim}
7
\end{verbatim}


Assume that \texttt{inc} and \texttt{dec} are primitives available that increment
and decrement the argument respectively. How is the procedure \texttt{pa+}
working? Let's apply the substitution rule.

\begin{minted}[]{racket}
(pa+ 3 4)
(if (= 3 0)
    4
    (pa+ (dec 3) (inc 4)))
(pa+ 2 5)
...
(pa+ 1 6)
...
(pa+ 0 7)
7
\end{minted}

We're skipping some steps, but the idea is that \texttt{x} keeps giving
one ``unit'' to \texttt{y} until it reaches zero. Then the sum is \texttt{y}.
Written with steps skipped:

\begin{minted}[]{racket}
(pa+ 3 4)
(pa+ 2 5)
(pa+ 1 6)
(pa+ 0 7)
7
\end{minted}

\subsection{Another Peano Adder}
\label{sec:org4afef1a}
Consider:
\begin{minted}[]{racket}
(define (pb+ x y)
  (if (= x 0)
      y
      (inc (pb+ (dec x) y))))
\end{minted}


This is also a Peano adder: but it's implemented \emph{slightly}
differently syntax-wise, a few characters here and there. Let's
use the substitution rule to see how it works.

\begin{minted}[]{racket}
(pb+ 3 4)
(inc (pb+ 2 4))
(inc (inc (pb+ 1 4)))
(inc (inc (inc (pb+ 0 4))))
(inc (inc ((inc 4))))
(inc (inc 5))
(inc 6)
7
\end{minted}

See that it \emph{does} work:

\begin{minted}[]{racket}

(pb+ 3 4)
\end{minted}

\begin{verbatim}
7
\end{verbatim}


Now, consider how these two, \texttt{pa+} and \texttt{pb+}, are different. While
the \emph{procedures} do the same thing, the processes are wildly
different. Let's discuss their time and space complexity.
It should be obvious to you that the time complexity is the
vertical axis in the substitution rule application, since the
interpreter ``executes'' these instructions line by line. More lines
means more time.

In the case of \texttt{pa+}, the number of lines increases by 1 if you
increase input \texttt{x} by 1. Thus, the time complexity is \(O(x)\).
Similarly, in the case of \texttt{pb+}, the number of lines increases by
2 (once in the expansion, once in the contraction) when you
increase \texttt{x} by 1. Thus, it is also \(O(x)\).

Now, the horizontal axis shows us how much space is being used. In
the case of \texttt{pa+}, the space used is a constant. Thus, \(O(1)\). On
the other hand, see that \texttt{pb+} first \emph{expands} then \emph{contracts}.
The length of the maximum expansion increases by 1 if we increase
\(x\) by 1, since there's one more increment to do. Thus, \(O(x)\).

Now, we call a process like \texttt{pa+} \emph{linear iterative} and a process
like \texttt{pb+} \emph{linear recursive}.

\begin{center}
\begin{tabular}{lccl}
\toprule
Process & Time Complexity & Space Complexity & Type\\
\midrule
\texttt{pa+} & \(O(x)\) & \(O(1)\) & Linear iterative\\
\texttt{pb+} & \(O(x)\) & \(O(x)\) & Linear recursive\\
\bottomrule
\end{tabular}
\end{center}

Note that the \emph{process} \texttt{pa+} being iterative has nothing to do
with the implementation/definition of the \emph{procedure}, which is
recursive. Iteration refers to the constant space requirement.

\section{Differentiating Between Iterative and Recursive Processes}
\label{sec:org34142f5}

One of the primary ways to differentiate between an iterative and
recursive process is to imagine what'd happen if you turned the
computer off, then resumed the current computation.

In a recursive process, we've lost some important information: how
deep into the recursion we are. In the \texttt{pb+} example, we wouldn't
know how many \texttt{inc}'s deep we are (information stored in the RAM by
the interpreter, not by the process), meaning that we can't return
the right value.

In an iterative process, we can pick up right where we left off,
since \emph{all} state information is contained by the process.

\section{Fibonacci Numbers}
\label{sec:org756bebe}

Fibonacci numbers are defined as:

$$F(x) =
   \begin{cases}
   0, & x = 0\\
   1, & x = 1\\
   F(x-1) + F(x-2), & \mathrm{otherwise}
   \end{cases}$$

The series itself is:
$$0,1,1,2,3,5,8,13,21,34,55\hdots$$

Let's write a Lisp function to calculate the \(n\mathrm{th}\) Fibonacci
number, assuming 0 is the 0th.

\begin{minted}[]{racket}
(define (fib n)
  (if (< n 2)
      n
      (+ (fib (- n 1))
         (fib (- n 2)))))
(fib 10)
\end{minted}

\begin{verbatim}
55
\end{verbatim}


It works, that's true. But how \emph{well} does it work. Let's see. When
we call (say) \texttt{(fib 4)}, we also call \texttt{(fib 3)} and \texttt{(fib 2)}, both
of which also call \(\hdots\) let's draw it:

\begin{center}
\begin{forest}
[\texttt{(fib 4)}
[\texttt{(fib 3)}
[\texttt{(fib 2)} [\texttt{(fib 1)} [1]] [\texttt{(fib 0)} [0]]]
[\texttt{(fib 1)} [1]]]
[\texttt{(fib 2)} [\texttt{(fib 1)} [1]] [\texttt{(fib 0)} [0]]]]
\end{forest}
\end{center}

A tree! Clearly, this is an exponential-time process, since
computing \(n+1\) takes exponentially more effort. Also note that
it's a pretty bad process, since we constantly recompute many
values. The space complexity is the maximum depth of the tree
(depth of recursion), which is at most \(n\). Therefore, the time
complexity is \(O(\mathrm{fib}(n))\) and space complexity is \(O(n)\).

It is useful to try and write an iterative Fibonacci with better
performance as an exercise.

\section{Towers of Hanoi}
\label{sec:org187fec9}

From Wikipedia:

\begin{quote}
The Tower of Hanoi is a mathematical game or puzzle. It consists of
three rods and a number of disks of different diameters, which can
slide onto any rod. The puzzle starts with the disks stacked on one
rod in order of decreasing size, the smallest at the top, thus
approximating a conical shape. The objective of the puzzle is to
move the entire stack to the last rod, obeying the following simple
rules:

\begin{itemize}
\item Only one disk may be moved at a time.
\item Each move consists of taking the upper disk from one of the
stacks and placing it on top of another stack or an empty rod.
\item No disk may be placed on top of a disk that is smaller than it.
\end{itemize}
\end{quote}

Let's try to solve Hanoi for 4 disks, from rod A to rod C. Again
--- ``wishful thinking''. Let's assume that we know how to solve for
3 disks. To solve, we'd take the top 3 disks, put it on the spare
rod B. Then, we'd take the fourth and largest disk, and put it on
destination rod C. Finally, we'd move the three disk pile from B
to C. Solved!

But wait --- to solve the 3 disk case, let's assume we know how to
solve the 2 disk case.

To solve the 2 disk case, we should know how
to solve the one disk case, which is just moving a disk from a rod
to another.

Or, in Lisp,

\begin{minted}[]{racket}
(define (move n from to spare)
  (cond ((= n 1) (display "Move disk at rod ")
                 (display from)
                 (display " to rod ")
                 (display to)
                 (display ".\n"))
        (else
         (move (- n 1) from spare to)
         (move 1 from to spare)
         (move (- n 1) spare to from))))

(move 4 "A" "C" "B")
\end{minted}

\begin{verbatim}
Move disk at rod A to rod B.
Move disk at rod A to rod C.
Move disk at rod B to rod C.
Move disk at rod A to rod B.
Move disk at rod C to rod A.
Move disk at rod C to rod B.
Move disk at rod A to rod B.
Move disk at rod A to rod C.
Move disk at rod B to rod C.
Move disk at rod B to rod A.
Move disk at rod C to rod A.
Move disk at rod B to rod C.
Move disk at rod A to rod B.
Move disk at rod A to rod C.
Move disk at rod B to rod C.
\end{verbatim}

Note, of course, that this procedure too, is an exponential time
procedure. However, any procedure for Hanoi will be exponential
time, since for \(n\) disks, Hanoi requires \(2^{n-1}\) moves. Even if
you compute every move in \(O(1)\) (which we do, since it's just a
print), the complexity will be \(O(2^n)\).

\section{Iterative Fibonacci}
\label{sec:org3af6e0e}

\begin{minted}[]{racket}
(define (iter-fib n a b)
  (if (= n 1)
      b
      (iter-fib (dec n) b (+ a b))))

(define (fib n)
  (iter-fib n 0 1))

(fib 10)
\end{minted}

\begin{verbatim}
55
\end{verbatim}

\chapter{Lecture 2A: Higher-Order Procedures}
\label{sec:org430aaee}

\section{Abstracting Procedural Ideas}
\label{sec:org8d6969f}

Consider the functions and their respective (recursive) procedures:

$$\sum_{i=a}^{b} i$$

\begin{minted}[]{racket}
(define (sum-int a b)
  (if (> a b)
      0
      (+ a
         (sum-int (inc a) b))))

(sum-int 0 10)
\end{minted}

\begin{verbatim}
55
\end{verbatim}


$$\sum_{i=a}^{b} i^{2}$$

\begin{minted}[]{racket}

(define (sum-sq a b)
  (if (> a b)
      0
      (+ (square a)
         (sum-sq (inc a) b))))

(sum-sq 0 4)
\end{minted}

\begin{verbatim}
30
\end{verbatim}


$$\sum_{i=a_{\mathrm{~by~}4}}^{b} \frac{1}{i(i+2)}$$

Note that this series estimates \(\pi /8\).

\begin{minted}[]{racket}
(define (sum-pi a b)
  (if (> a b)
      0
      (+ (/ 1
            (* a (+ a 2)))
         (sum-pi (+ a 4) b))))

(* 8 (sum-pi #i1 #i1000000))
\end{minted}

\begin{verbatim}
3.141590653589793
\end{verbatim}



See that the commonality between these procedures comes from the
fact that the notion of ``summation'' from \texttt{a} to \texttt{b} is the same,
but the \emph{function} being summed is different in each case. Or, in
general form:

\begin{minted}[]{racket}
(define (<name> a b)
  (if (> a b)
      0
      (+ (<term> a)
         (<name> (<next> a) b))))
\end{minted}

The way to solve this is by writing a procedure \texttt{sum}, which has
available to it two procedures \texttt{term} and \texttt{next}. We supply these
as arguments. Consider:

\begin{minted}[]{racket}
(define (sum term a next b)
  (if (> a b)
      0
      (+ (term a)
         (sum term (next a) next b))))
\end{minted}

When we call \texttt{sum} recursively, see that we pass to it the \emph{same
procedures} \texttt{term} and \texttt{next}, along with \texttt{b} and the next value of
\texttt{a}. Now, it is easy to define \texttt{sum-int}, \texttt{sum-sq}, and \texttt{sum-pi}
using \texttt{sum}, thus:

\begin{minted}[]{racket}

(define (sum-int a b)
  (define (identity x) x)
  (sum identity
       a
       inc
       b))

(sum-int 0 10)
\end{minted}

\begin{verbatim}
55
\end{verbatim}


\texttt{identity} is the function \(p(x) = x\).

\begin{minted}[]{racket}


(define (sum-sq a b)
  (sum square
       a
       inc
       b))

(sum-sq 0 4)
\end{minted}

\begin{verbatim}
30
\end{verbatim}


\begin{minted}[]{racket}

(define (sum-pi a b)
  (sum (lambda (x)
         (/ 1
            (* x (+ x 2))))
       a
       (lambda (x) (+ x 4))
       b))

(* 8 (sum-pi #i1 #i1000000))
\end{minted}

\begin{verbatim}
3.141590653589793
\end{verbatim}


Recall that \texttt{lambda} means ``make a procedure'' that is nameless. In
\texttt{sum-pi}, we choose to give it anonymous functions as arguments
instead of defining our own, because there's no reason to name a
procedure we won't later use.

The big advantage of abstracting away \texttt{sum} this way is that in
case we want to implement it in a different way, we merely have to
change the implementation of one function (\texttt{sum}) and not that of
the three functions that use it. In fact, those functions can
remain exactly the same.

Here's another implementation of \texttt{sum}. See that \texttt{sum-pi} still
works without changes, because it doesn't care about how \texttt{sum} is
implemented as long as the argument number and order remains
constant.

\begin{minted}[]{racket}
(define (sum term a next b)
  (define (iter j ans)
    (if (> j b)
        ans
        (iter (next j)
              (+ (term j)
                 ans))))
  (iter a 0))

(define (sum-pi a b)
  (sum (lambda (x)
         (/ 1
            (* x (+ x 2))))
       a
       (lambda (x) (+ x 4))
       b))

(* 8 (sum-pi #i1 #i1000000))
\end{minted}

\begin{verbatim}
3.1415906535898936
\end{verbatim}

\section{More on Square Roots}
\label{sec:orgd64592f}

Recall our square root procedure. When seen in Lisp code, it's not
very clear what it's doing, or how it's working.

\begin{minted}[]{racket}
(define (sqrt x)
  (define (good-enough-p g)
    (define (square g)
      (* g g))
    (define (abs y)
      (if (< y 0)
          (- y)
          y))
    (< (abs (- (square g) x))
       0.0001))
  (define (improve g)
    (define (average y z)
      (/ (+ y z) 2))
    (average g (/ x g)))
  (define (try g)
    (if (good-enough-p g)
        g
        (try (improve g))))
  (try 1))
\end{minted}

\begin{minted}[]{racket}

(sqrt #i2)
\end{minted}

\begin{verbatim}
1.4142156862745097
\end{verbatim}


Let's use higher-order procedure abstraction to make it clearer.

\subsection{Fixed Points}
\label{sec:org3185626}

Recall that the algorithm itself relies on writing a function

$$f\colon y\mapsto \frac{y+\frac{x}{y}}{2}$$

Note that this works because \(f(\sqrt{x}) = \sqrt{x}\):

$$f(\sqrt{x})=\frac{\sqrt{x}+\frac{x}{\sqrt{x}}}{2} = \frac{2\sqrt{x}}{2} = \sqrt{x}$$

See that this is \emph{actually} an algorithm for finding a fixed point
of a function \(f\), which is defined as finding the point where
\(f(z)=z\). This algorithm is merely an instance of a function \(f\)
whose fixed point happens to be the square root.

\begin{quote}
For some functions, the fixed point can be found by iterating it.
\end{quote}

This is the top-level abstraction we'll write a function for.
First, let's see how we'd write a square-root function by wishful
thinking:

\begin{minted}[]{racket}

(define (sqrt x)
  (fixed-point
   (lambda (y) (average (/ x y)
                        y))
   1))
\end{minted}

Now writing \texttt{fixed-point}:

\begin{minted}[]{racket}

(define (fixed-point f start)
  (define (close-enough-p x y)
    (< (abs (- x y))
       0.00001))
  (define (iter old new)
    (if (close-enough-p old new)
        new
        (iter new (f new))))
  (iter start (f start)))
\end{minted}

Let's try it out!

\begin{minted}[]{racket}


(sqrt #i2)
\end{minted}

\begin{verbatim}
1.4142135623746899
\end{verbatim}

\subsection{Damping Oscillations}
\label{sec:orge5ac4ef}

A fair question when seeing the function
$$f_1\colon y\mapsto \frac{y+\frac{x}{y}}{2}$$
is why another function
$$f\colon y\mapsto \frac{x}{y}$$
wouldn't work in its place. This question is best
answered by trying to find its fixed point by iteration. Let's try
to find it for \(x=2\), starting at \(y=1\). Then,

$$f(1) = \frac{2}{1} = 2$$
$$f(2) = \frac{2}{2} = 1$$
$$f(1) = \frac{2}{1} = 2$$
$$f(2) = \frac{2}{2} = 1$$
$$~\hdots$$

It seems that instead of converging, this function is
\emph{oscillating} between two values. We know that it's easy to fix
this: we have to damp these oscillations. The most natural way to
do this is to take the average of successive values \(y\) and
\(f(y)\). A \texttt{sqrt} function that uses average damping would be:

\begin{minted}[]{racket}

(define (sqrt x)
  (fixed-point
   (avg-damp (lambda (y) (/ x y)))
   1))
\end{minted}

The \texttt{avg-damp} function takes in a procedure, creates an average damping
procedure, and returns it. Or, in Lisp:

\begin{minted}[]{racket}

(define avg-damp
  (lambda (f)
    (lambda (x) (average (f x) x))))
\end{minted}

It is worth discussing how \texttt{avg-damp} works. It is defined as a
procedure which takes the argument of a function \texttt{f}. It then
returns an anonymous procedure which takes an argument \texttt{x}, and
computes the average of \(f(x)\) and \(x\). This is finally the
highest level of abstraction we can reach for the \texttt{sqrt}
algorithm --- finding the fixed point of a damped oscillating
function.

Using the \texttt{sqrt} function,

\begin{minted}[]{racket}


(sqrt #i2)
\end{minted}

\begin{verbatim}
1.4142135623746899
\end{verbatim}

\section{Newton's Method}
\label{sec:org0795127}

Newton's method is used to find the zeros of a function (\(y \ni
   f(y)=0\)). To use it, start with some guess \(y_0\). Then,

$$y_{n+1} = y_n - \frac{f(y_n)}{f'(y_n)}$$

where $$f'(y) = \frac{\mathrm{d}f(y)}{\mathrm{d}y}$$

We can, of course, find the zero of the square root finding function
\(f(y) =  x-y^2\) using Newton's method. Note that Newton's method
\emph{itself} is based on fixed points, since it aims to find a fixed
point where \(y_{n+1}\approx y_n\).

Defining \texttt{sqrt}:

\begin{minted}[]{racket}

(define (sqrt x)
  (newton (lambda (y) (- x (square y)))
          1))
\end{minted}

We pass to \texttt{newton} a function \(f(y)=x-y^2\), since its zero is \(x=y^2\).

\begin{minted}[]{racket}

(define (newton f guess)
  (define df (deriv f))
  (fixed-point
   (lambda (x) (- x
                  (/ (f x)
                     (df x))))
   guess))
\end{minted}


It is important to note that defining \texttt{df} to be \texttt{(deriv f)} once
prevents wasteful recomputation of \texttt{df} every time \texttt{fixed-point}
calls itself.

Of course, we now have to define a derivative function. We can
simply use the standard limit definition to find it numerically:

$$f'(x) = \lim_{\Delta x\to 0} \frac{f(x+\Delta x) - f(x)}{\Delta
   x}$$

Or, in Lisp,

\begin{minted}[]{racket}
(define dx 0.0000001)

(define deriv
  (lambda (f)
    (lambda (x)
      (/ (- (f (+ x dx))
            (f x))
         dx))))


\end{minted}

This function returns a function which is the derivative of \texttt{f},
and can be used as such. Consider:

\begin{minted}[]{racket}

((deriv (lambda (x) (* x x x))) 2)
\end{minted}

\begin{verbatim}
12.000000584322379
\end{verbatim}


Which is the expected value of differentiating \(x^{3}\) w.r.t \(x\)
(\(3x^2\)) and evaluating at 2.

Testing out our \texttt{sqrt} function:

\begin{minted}[]{racket}



(sqrt #i2)
\end{minted}

\begin{verbatim}
1.4142135623747674
\end{verbatim}

\section{Procedures are First-Class Citizens}
\label{sec:orgf7f1754}

This means that procedures can be:
\begin{itemize}
\item Named using variables.
\item Passed as arguments to procedures.
\item Returned as values from procedures.
\item Included in data structures.
\end{itemize}

\chapter{Lecture 2B: Compound Data}
\label{sec:org0cdec26}

Consider our \texttt{sqrt} function that uses \texttt{good-enough-p}. What we did
while writing \texttt{sqrt} is assume the existence of \texttt{good-enough-p}.
That is, we divorced the task of building \texttt{sqrt} from the task of
implementing its parts.

Let's do this for data.

\section{Rational Number Arithmetic}
\label{sec:org4aff986}

Let's design a system which can add fractions:
$$\frac{1}{2}+\frac{1}{4}=\frac{3}{4}$$
and multiply them:
$$\frac{3}{4}\times \frac{2}{3} = \frac{1}{2}$$

The \emph{procedures} for these two tasks are well known to most people:

$$\frac{n_1}{d_1} + \frac{n_2}{d_2} = \frac{n_1d_2+n_2d_2}{d_1d_2}$$
and
$$\frac{n_1}{d_1} \times \frac{n_2}{d_2} = \frac{n_1n_2}{d_1d_2}$$

\subsection{Abstraction}
\label{sec:org87c0221}
We don't know, however, how to represent this data in a Lisp
procedure. Let's use our powerful ``wishful thinking'' strategy.
Assume that we have the following procedures available to us:

\begin{itemize}
\item A constructor \texttt{(make-rat n d)} which makes a fraction with
numerator \texttt{n} and denominator \texttt{d}.
\item Two selectors:
\begin{itemize}
\item \texttt{(numer x)} which takes in a fraction \texttt{x} and returns its
numerator.
\item \texttt{(denom x)} which takes in a fraction \texttt{x} and returns its
denominator.
\end{itemize}
\end{itemize}

Then, our procedures are easy to write:

\begin{minted}[]{racket}
(define (+rat x y)
  (make-rat
   (+ (* (numer x) (denom y))
      (* (numer y) (denom x)))
   (* (denom x) (denom y))))

(define (*rat x y)
  (make-rat
   (* (numer x) (numer y))
   (* (denom x) (denom y))))
\end{minted}

Why do we need this data object abstraction anyway? We could very
well define \texttt{+rat} to take in four numbers, two numerators and two
denominators. But to return, we can't return \emph{both} numerator and
denominator. We now have to define two summation functions, one for
the numerator and one for the denominator, and somehow keep track
of the fact that one of these number is the numerator and the other
the denominator. Furthermore, when applying more complex operations
like:

\begin{minted}[]{racket}
(*rat (+rat x y)
      (+rat s t))
\end{minted}

The data abstraction helps. If it weren't there, we'd have to
maintain some temporary registers to store the numerator and
denominator values of the \texttt{+rat} operations into, then pass them to
\texttt{*rat}.

Worse than confusing the program, such a design philosophy would
confuse us, the programmers.

\subsection{Data Object Creation}
\label{sec:org3ff46b7}

The glue we use to stick two numbers together is provided by three
Lisp primitives:
\begin{itemize}
\item A constructor \texttt{cons}, which generates an ordered pair.
\item Two selectors:
\begin{itemize}
\item \texttt{car}, which selects the first element of the pair, and
\item \texttt{cdr}, which selects the second element of the pair.
\end{itemize}
\end{itemize}

In use,
\begin{minted}[]{racket}
(define x (cons 1 2))
(car x)
(cdr x)
\end{minted}

\begin{verbatim}
1
2
\end{verbatim}


We can now write the procedures that we'd deferred writing
earlier:

\begin{minted}[]{racket}
(define (make-rat x y)
  (cons x y))

(define (numer x)
  (car x))

(define (denom x)
  (cdr x))
\end{minted}

\begin{minted}[]{racket}



(define x (make-rat 1 2))
(define y (make-rat 1 4))
(define z (+rat x y))
(numer z)
(denom z)
\end{minted}

\begin{verbatim}
6
8
\end{verbatim}


Agh. We forgot to reduce results to the simplest form. We can
easily include this in the \texttt{make-rat} procedure:\footnote{\texttt{let} is a Lisp primitive which takes as its first argument a
list of definitions, and second input a list of applications that may
use these definitions. The trick is that these definitions are only
valid in the body (second argument) of \texttt{let}, effectively creating a
local namespace.}

\begin{minted}[]{racket}
(define (make-rat x y)
  (let ((g (gcd x y)))
    (cons (/ x g)
          (/ y g))))

(define (numer x)
  (car x))

(define (denom x)
  (cdr x))
\end{minted}

Note that we could shift the \texttt{gcd} bit to functions \texttt{numer} and
\texttt{denom}, which would display the simplest form at access time
rather than creation time. Deciding between the two is a matter of
system efficiency: a system which displays often should use
creation time simplification, while a system which creates many
fractions should use access time simplification.
We now need a GCD function:

\begin{minted}[]{racket}
(define (gcd a b)
  (if (= b 0)
      a
      (gcd b (remainder a b))))
\end{minted}

We can now use \texttt{+rat} in \emph{exactly} the same way, since the
interface is the same. This is the advantage of abstraction.

\begin{minted}[]{racket}



(define x (make-rat 1 2))
(define y (make-rat 1 4))
(define z (+rat x y))
(numer z)
(denom z)
\end{minted}

\begin{verbatim}
3
4
\end{verbatim}


Excellent: we now have a working system. The data abstraction
model can be visualised as follows:

\begin{center}
\rule{6cm}{2pt}\\
\texttt{+rat}, \texttt{*rat} \ldots{}\\
\rule{6cm}{2pt}\\
\texttt{make-rat}, \texttt{numer}, \texttt{denom}\\
\rule{6cm}{2pt}\\
\texttt{gcd}\\
\rule{6cm}{2pt}\\
Pairs\\
\rule{6cm}{2pt}
\end{center}

At each layer of abstraction, we merely care about the usage of
the lower layers and not their implementation or underlying
representation.

\section{Representing Points on a Plane}
\label{sec:org0440dc7}

This is now an easy problem --- the code should be
self-explanatory.

\begin{minted}[]{racket}
(define (make-vec x y)
  (cons x y))

(define (xcor v)
  (car v))

(define (ycor v)
  (cdr v))
\end{minted}

We could now define a segment as a pair of vectors:

\begin{minted}[]{racket}
(define (make-seg v w)
  (cons v w))

(define (seg-start s)
  (car s))

(define (seg-end s)
  (cdr s))
\end{minted}

Some sample operations:

\begin{minted}[]{racket}






(define (midpoint s)
  (let ((a (seg-start s))
        (b (seg-end s)))
    (make-vec
     (average (xcor a) (xcor b))
     (average (ycor a) (ycor b)))))

(define (length s)
  (let ((dx (- (xcor (seg-end s))
               (xcor (seg-start s))))
        (dy (- (ycor (seg-end s))
               (ycor (seg-start s)))))
    (sqrt (+ (square dx)
             (square dy)))))

(define side-a (make-vec #i3 #i0))
(define side-b (make-vec #i0 #i4))
(define segment (make-seg side-a side-b))

(length segment)

(define mp (midpoint segment))

(xcor mp)
(ycor mp)
\end{minted}

\begin{verbatim}
5.000000000053722
1.5
2.0
\end{verbatim}


The abstraction layer diagram of this code is:
\begin{center}
\rule{6cm}{2pt}\\
Segments\\
\rule{6cm}{2pt}\\
Vectors\\
\rule{6cm}{2pt}\\
Pairs\\
\rule{6cm}{2pt}
\end{center}

It is interesting to note that segments are pairs of vectors,
which are pairs of numbers, so segments are actually pairs of
pairs. Represented as a tree:

\begin{center}
\begin{forest}
[$s$ [$\vec{v_{1}}$ [$x_{1}$] [$y_{1}$]] [$\vec{v_{2}}$ [$x_2$] [$y_2$]]]
\end{forest}
\end{center}

This property is called \emph{closure} (from abstract algebra\footnote{For an operation defined on members of a set, the result of
that operation is a member of the set. For instance, addition on
natural numbers.}): that means
of combination can be nested recursively. It's an important and
powerful technique.

For instance, a three-dimensional vector can be represented by a
pair whose one element is a number and whose other element is a
pair of numbers. Or, in Lisp:

\begin{minted}[]{racket}
(define three-d-vec (cons 3 (cons 4 5)))
(car three-d-vec)
(car (cdr three-d-vec))
(cdr (cdr three-d-vec))
\end{minted}

\begin{verbatim}
3
4
5
\end{verbatim}

\section{Pairs}
\label{sec:orgce5110b}

Let's go back to when we assumed that \texttt{make-rat}, \texttt{numer}, and
\texttt{denom}, were already implemented. The procedures we then wrote
were written using \emph{abstract data}, with the only ``assured''
property being that:

\begin{verse}
\texttt{if x = (make-rat n d):}\\
\vspace*{1em}
\hspace*{2em}\(\displaystyle \frac{\mathtt{numer~x}}{\mathtt{denom~x}} = \frac{\mathtt{n}}{\mathtt{d}}\)\\
\end{verse}

Beyond this basic ``spec'', or the interface contract, we know
nothing about its implementation.

Now, it's easy not to appreciate how knowing \emph{merely} the
specification of the layer below is sufficient to use it, so let's
discuss how pairs work. When we wanted to implement \texttt{make-rat}, we
kind of ``cheated'' in that we said, ``Okay, Lisp has a primitive to
do this so we don't have to implement a pair.'' Let's now take a
look at a possible implementation of a pair that doesn't use data
objects at all, and instead mimics them from thin air. Consider:

\begin{minted}[]{racket}
(define (our-cons a b)
  (lambda (pick)
    (cond ((= pick 1) a)
          ((= pick 2) b))))

(define (our-car x) (x 1))
(define (our-cdr x) (x 2))
\end{minted}

\begin{minted}[]{racket}

(define pair (our-cons 3 4))
(our-car pair)
(our-cdr pair)
\end{minted}

\begin{verbatim}
3
4
\end{verbatim}


Before thinking about how it works: consider the fact that Lisp's
pairs could be implemented this way, and not only would we not know
about this while implementing \texttt{make-rat} --- we wouldn't care,
since it's below the level of abstraction we're working at. As long
as it behaves the way we expect it to --- that is, it follows the
``spec'', we don't know or care about its implementation\footnote{Note that Lisp actually implements pairs using ``real'' data
structures, since using procedures this way is less efficient.}. Such is the
power of abstraction.

Now, how is this implementation even working? Well:
\begin{itemize}
\item \texttt{cons} is a procedure that returns a lambda (anonymous procedure)
which, by the substitution model, looks like:
\begin{minted}[]{racket}
(lambda (pick)
  (cond ((= pick 1) 3)
        ((= pick 2) 4)))
\end{minted}
\item \texttt{car} expects this procedure as an input, and returns the result of
supplying this procedure with the value 1. This is naturally the
first of the two numbers given to \texttt{cons} (\texttt{a}).
\item \texttt{cdr} is identical to \texttt{car}, except that \emph{it} supplies the input
procedure with argument 2 to get \texttt{b}.
\end{itemize}

We can thus implement a pair ``data structure'' using only lambdas.
In fact, these pairs are closed:

\begin{minted}[]{racket}

(define three-d-vec (our-cons 3 (our-cons 4 5)))
(our-car three-d-vec)
(our-car (our-cdr three-d-vec))
(our-cdr (our-cdr three-d-vec))
(our-cdr three-d-vec)
\end{minted}

\begin{verbatim}
3
4
5
#<procedure:...6f_i/ob-2136OZJ.rkt:4:2>
\end{verbatim}


It is worth thinking about the structure of \texttt{three-d-vec}:
\begin{minted}[]{racket}
(lambda (pick)
  (cond ((= pick 1) 3)
        ((= pick 2) (lambda (pick)
                      (cond ((= pick 1) 4)
                            ((= pick 2) 5))))))
\end{minted}

Picking \texttt{2} in the top-level lambda gives us another lambda, in
which we can pick either the first number (4) or the second (5).
Note that this is precisely the nested pair structure we were going
for.

\begin{center}
\begin{forest}
[$\lambda$(p) [3] [$\lambda$(p) [4] [5]]]
\end{forest}
\end{center}

\chapter{Lecture 3A: Henderson Escher Example}
\label{sec:org40383f9}

Recall our vector procedures:

\begin{minted}[]{racket}
(define (make-vec x y)
  (cons x y))

(define (xcor v)
  (car v))

(define (ycor v)
  (cdr v))
\end{minted}

We could define more procedures using these:

\begin{minted}[]{racket}
(define (+vect v1 v2)
  (make-vec
   (+ (xcor v1) (xcor v2))
   (+ (ycor v1) (ycor v2))))

(define (scale v s)
  (make-vec
   (* s (xcor v))
   (* s (ycor v))))
\end{minted}

Recall that our representation of a line segment was as a pair of
vectors, or pair of pairs. That is, we can use the property of
closure that pairs have to store any amount of data.

\section{Lists}
\label{sec:org0d51808}
Often, we want to store a sequence of data. Using pairs, there are
many ways to do this, for instance:

\begin{minted}[]{racket}
(cons (cons 1 2) (cons 3 4))
(cons (cons 1 (cons 2 3)) 4)
\end{minted}

\begin{verbatim}
((1 . 2) 3 . 4)
((1 2 . 3) . 4)
\end{verbatim}


However, we want to establish a conventional way of dealing with
sequences, to prevent having to make ad-hoc choices. Lisp uses a
representation called a list:

\begin{minted}[]{racket}
(cons 1 (cons 2 (cons 3 (cons 4 nil))))
\end{minted}

\begin{verbatim}
(1 2 3 4)
\end{verbatim}


Note that the \texttt{nil} represents the null or empty list. Since
writing so many \texttt{cons} is painful, Lisp provides the primitive
\texttt{list} which lets us build such a structure.

\begin{minted}[]{racket}
(list 1 2 3 4)
\end{minted}

\begin{verbatim}
(1 2 3 4)
\end{verbatim}


Note that \texttt{list} is merely syntactic sugar for building up using
pairs:

\begin{minted}[]{racket}
(define one-to-four (list 1 2 3 4))
\end{minted}

\begin{minted}[]{racket}

(car one-to-four)
(cdr one-to-four)
(car (cdr one-to-four))
(cdr (cdr one-to-four))
(car (cdr (cdr (cdr one-to-four))))
(cdr (cdr (cdr (cdr one-to-four))))
\end{minted}

\begin{verbatim}
1
(2 3 4)
2
(3 4)
4
()
\end{verbatim}


Note that the empty list, \texttt{nil}, is also represented by \texttt{()}. This
way of walking down the list for elements is called \texttt{cdr}-ing down
a list, but it's a bit painful. Thus, when we want to process
lists, we write procedures.

\subsection{Procedures on Lists}
\label{sec:org150b893}

Say we wanted to write a procedure \texttt{scale-list} which multiplies
every element in the list by a certain value. That is, when scale
list is called on \texttt{one-to-four} with value 10, it returns \texttt{(10 20
    30 40)}. Here's one possible (recursive) implementation:

\begin{minted}[]{racket}

(define (scale-list l scale)
  (if (null? l)
      nil
      (cons (* scale (car l))
            (scale-list (cdr l) scale))))

(scale-list one-to-four 10)
\end{minted}

\begin{verbatim}
(10 20 30 40)
\end{verbatim}


\texttt{null?} is a predicate which tells us whether the given input is
the empty list. This will be the case at the end of the list.
Of course, this is \emph{actually} a general method for processing all
values of a list and returning another list, so we write a
higher-order procedure which applies a procedure to all elements
of a list and returns the result as a list, called \texttt{map}.

\begin{minted}[]{racket}
(define (map p l)
  (if (null? l)
      nil
      (cons (p (car l))
            (map p (cdr l)))))
\end{minted}

Now defining \texttt{scale-list} in terms of \texttt{map}:

\begin{minted}[]{racket}


(define (scale-list l s)
  (map (lambda (x) (* x s))
       l))

(scale-list one-to-four 20)
\end{minted}

\begin{verbatim}
(20 40 60 80)
\end{verbatim}


We can now square lists:
\begin{minted}[]{racket}



(map square one-to-four)
\end{minted}

\begin{verbatim}
(1 4 9 16)
\end{verbatim}


Similar to \texttt{map}, we define a higher-order procedure \texttt{for-each},
which, instead of \texttt{cons}-ing a list and returning it, simply
applies to procedure to each element of the list.

\begin{minted}[]{racket}
(define (for-each proc l)
  (cond ((null? l) done)
        (else
         (proc (car l))
         (for-each proc (cdr l)))))
\end{minted}

\section{Henderson's Picture Language}
\label{sec:org5ac88be}

Let's define a language. As usual, we'll concern ourselves with its
primitives, means of combination, and means of abstraction,
implementing some of this language in Lisp along the way.


\subsection{Primitives}
\label{sec:org1158fe8}
This language has only one primitive: ``picture'', which is a figure
scaled to fit a rectangle.


\subsection{Means of Combination and Operations}
\label{sec:org72ddb8b}

\begin{itemize}
\item Rotate, which rotates a picture and returns it.
\item Flip, which flips the picture across an axis and returns it.
\item Beside, which takes two pictures and a scale, then puts the two
next to each other, returning a picture.
\item Above, like beside, but above.
\end{itemize}

See that the closure property (that an operation on pictures
returns a picture)\footnote{\(p \otimes p = p\)} allows us to combine these operations/means of
combination to build complex pictures with ease.

Let's now implement this part of the language.

\subsection{An Implementation}
\label{sec:org405549e}

\begin{enumerate}
\item Rectangles
\label{sec:org61b515f}

Three vectors are needed to uniquely identify a rectangle on the
plane. By convention, we take these to be the bottom left corner
(``origin''), the bottom right corner (``horizontal'') and the top
left corner (``vertical''). Their positions can be described
relative to the \((0,0)\) of the display screen. Therefore,
rectangle is implemented by:
\begin{itemize}
\item Constructor \texttt{make-rect}.
\item Selectors \texttt{origin}, \texttt{horiz}, and \texttt{vert}, for the three vectors.
\end{itemize}

Note that technically, a rectangle describes a transformation of
the unit square, where each point in the unit square:
$$(x,y)\mapsto \mathtt{origin} + x\cdot \mathtt{horiz} + y\cdot
     \mathtt{vert}$$

We can define a procedure which returns a procedure which maps
a pair of points \((x,y)\) on the unit square to a given rectangle:

\begin{minted}[]{racket}
(define (coord-map rect)
  (lambda (point)
    (+vect
     (+vect (scale (xcor point)
                   (horiz rect))
            (scale (ycor point)
                   (vert rect)))
     (origin rect))))
\end{minted}

\texttt{coord-map} returns a procedure which given a point will map it
correctly to \texttt{rect}.

\item Pictures
\label{sec:org1e3831d}

We can now easily define a procedure which makes a picture:
\begin{minted}[]{racket}
(define (make-picture seglist)
  (lambda (rect)
    (for-each
     (lambda (s)
       (drawline
        ((coord-map rect) (seg-start s))
        ((coord-map rect) (seg-end s))))
     seglist)))
\end{minted}

Well, relatively easily. Let's explain what \texttt{make-picture}
actually does:

\begin{itemize}
\item Takes argument \texttt{seglist}, which is a list of line segments
(pairs of vectors) that the picture is.
\item Returns a procedure which:
\begin{itemize}
\item Takes the argument of a rectangle.
\item For every element in \texttt{seglist}:
\begin{itemize}
\item Draws the segment within rectangle, by scaling it correctly
using \texttt{coord-map}.
\item This is done by giving \texttt{coord-map} the rectangle to scale
to.
\item The procedure returned by \texttt{coord-map} then scales the
vectors \texttt{(seg-start s)} and \texttt{(seg-end s)} to the rectangle.
\item This can now be drawn by \texttt{drawline}, since it has as
arguments two points.
\end{itemize}
\end{itemize}
\end{itemize}

Note that a picture is \emph{actually} a procedure which draws itself
inside a given rectangle, and \texttt{make-picture} generates this
procedure from a \texttt{seglist}. Or, in use:

\begin{minted}[]{racket}
(define R (make-rect ;some vectors
           ))
(define draw-george-in-rectangle (make-picture ;some seglist
                ))
(draw-george-in-rectangle R)
\end{minted}

\item Beside
\label{sec:org86cedc7}

\texttt{beside} needs to draw two pictures on the screen, after scaling
them correctly (by \texttt{a}) and placing them side by side. Thus,
\texttt{beside} returns a picture which takes in an argument \texttt{rect}.
\texttt{beside} starts drawing the left picture at \texttt{(origin rect),
     (scale a (horiz rect)) (vert rect)} and the right picture at
\texttt{(+vect (origin rect) (scale a (horiz rect))), (scale (- 1 a)
     (horiz rect)), (vert rect)}. This places the two pictures side by
side and scales them correctly within \texttt{rect}. Or, in Lisp,

\begin{minted}[]{racket}
(define (beside p1 p2 a)
  (lambda (rect)
    (p1 (make-rect
         (origin rect)
         (scale a (horiz rect))
         (vert rect)))
    (p2 (make-rect
         (+vect (origin rect)
                (scale a (horiz rect)))
         (scale (-1 a) (horiz rect))
         (vert rect)))))
\end{minted}

\item Rotate-90
\label{sec:org1ae9e42}

To rotate a picture by 90 degrees counter-clockwise, all we have
to do is make the \texttt{origin} shift to where \texttt{horiz} is, then draw
the new \texttt{horiz} and \texttt{vert} correctly. With some vector algebra,
the procedure in Lisp is:

\begin{minted}[]{racket}
(define (rot90 pict)
  (lambda (rect)
    (pict (make-rect
           (+vect (origin rect)
                  (horiz rect))
           (vert rect)
           (scale -1 (horiz rect))))))
\end{minted}
\end{enumerate}


\subsection{Means of Abstraction}
\label{sec:org123e939}
See that the picture language is now embedded in Lisp. We can
write recursive procedures to modify a picture:

\begin{minted}[]{racket}
(define (right-push pict n a)
  (if (= n 0)
      pict
      (beside pict
              (right-push pict (dec n) a)
              a)))
\end{minted}

We can even write a higher order procedure for ``pushing'':
\begin{minted}[]{racket}
(define (push comb)
  (lambda (pict n a)
    ((repeated
      (lambda (p)
        (comb pict p a))
      n)
     pict)))

(define right-push (push beside))
\end{minted}

There's a lot to learn from this example:
\begin{itemize}
\item We're embedding a language inside Lisp. All of Lisp's power is
available to this small language now: including recursion.
\item There's no difference between a procedure and data: we're
passing pictures around exactly like data, even though it's
actually a procedure.
\item We've created a layered system of abstractions on top of Lisp,
which allows \emph{each layer} to have all of Lisp's expressive
power. This is contrasted to a designing such a system bottom-up
as a tree, which would mean that:
\begin{itemize}
\item Each node does a very specific purpose and is limited in
complexity because a new feature has to be built ground-up at
the node.
\item Making a change is near impossible, since there's no higher
order procedural abstraction. Making a change that affects
more than one node is a nightmare.
\end{itemize}
\end{itemize}

\chapter{Lecture 3B: Symbolic Differentiation; Quotation}
\label{sec:org0857fb7}

We saw that robust system design involves insensitivity to small
changes, and that embedding a language within Lisp allows this. Let
us turn to a somewhat similar thread, solving the problem of
symbolic differentiation in Lisp.

This problem is somewhat different from \emph{numerical} differentiation
of a function like we did for Newton's method, since we actually
want the expressions we work with to be in an algebraic language.
Before figuring out how to implement such a thing, let's talk about
the operation of differentiation itself.

\section{Differentiation v/s Integration}
\label{sec:org29733c1}

Why is it so much easier to differentiate than to integrate?
Let us look at the basic rules of differentiation:

$$\frac{\mathrm{d}k}{\mathrm{d}x} = 0$$
$$\frac{\mathrm{d}x}{\mathrm{d}x} = 1$$
$$\frac{\mathrm{d}k\cdot a}{\mathrm{d}x} = k\cdot \frac{\mathrm{d}a}{\mathrm{d}x}$$
$$\frac{\mathrm{d}(a+b)}{\mathrm{d}x} =
   \frac{\mathrm{d}a}{\mathrm{d}x} + \frac{\mathrm{d}b}{\mathrm{d}x}$$
$$\frac{\mathrm{d}(ab)}{\mathrm{d}x} =  a\cdot
   \frac{\mathrm{d}b}{\mathrm{d}x} +
   \frac{\mathrm{d}a}{\mathrm{d}x}\cdot b$$
$$\frac{\mathrm{d}x^{n}}{\mathrm{d}x} = nx^{n-1}$$

See that these rules are reduction rules, in that the derivative of
some complex thing is the derivative of simpler things joined
together by basic operations. Such reduction rules are naturally
recursive in nature. This makes the problem of differentiation very
easy to solve using simple algorithms.

On the other hand, implementing an integration system is a much
harder problem, since such a system would require us to go the
other way, combining up simpler expressions to make more
complicated ones, which often involves an intrinsically difficult
choice to make.

With these simple recursive rules in mind, let's implement a
symbolic differentiation system.

\section{Some Wishful Thinking}
\label{sec:orgdaad0fa}

\begin{minted}[]{racket}
(define (deriv expr var)
  (cond ((constant? expr var) 0)
        ((same-var? expr var) 1)
        ((sum? expr)
         (make-sum (deriv (a1 expr) var)
                   (deriv (a2 expr) var)))
        ((product? expr)
         (make-sum
          (make-product (m1 expr)
                        (deriv (m2 expr) var))
          (make-product (deriv (m1 expr) var)
                        (m2 expr))))))
\end{minted}


That's enough rules for now, we can add more later.

Note that \texttt{a1} is a procedure returning the first term of the
addition \(x+y\) (in this case, \(x\)), and \texttt{a2} is a procedure
returning the second (in this case, \(y\)). Similar for
multiplication, \texttt{m1} and \texttt{m2}.

All the -\texttt{?} procedures are predicates, and should be
self-explanatory. \texttt{make-}, as expected, makes the object with given
arguments as values and returns it. These are a level of
abstraction below \texttt{deriv}, and involve the actual representation of
algebraic expressions. Let's figure out how to do this.

\section{Representing Algebraic Expressions}
\label{sec:org14b7078}

\subsection{Using Lisp Syntax}
\label{sec:orgde1da44}

One very simple way to represent expressions is to use Lisp's way:
expressions that form trees. Consider:

$$ax^{2} \mapsto \mathtt{(*~a~(*~x~x))}$$ $$bx+c \mapsto \mathtt{(
    \mathtt{+} ~(*~b~x)~c)}$$

This has the advantage that representing such expression is just a
list. Moreover, finding out the operation is merely the \texttt{car} of
the list, and the operands are the \texttt{cdr}. This effectively
eliminates our need for parsing algebraic expressions.

\subsection{Representation Implementation}
\label{sec:org65c770d}

Let's start defining our procedures.

\begin{minted}[]{racket}
(define (constant? expr var)
  (and (atom? expr)
       (not (eq? expr var))))

(define (same-var? expr var)
  (and (atom? expr)
       (eq? expr var)))

(define (sum? expr)
  (and (not (atom? expr))
       (eq? (car expr) '+)))

(define (product? expr)
  (and (not (atom? expr))
       (eq? (car expr) '*)))
\end{minted}

We see a new form here: \texttt{'+} and \texttt{'*}. This is called ``quoting''.
Why do we need to do this? Consider:

\begin{verse}
``Say your name!''\\
``Susanne.''\\
``Say `your name'!''\\
``Your name.''\\
\end{verse}

To differentiate the cases where we mean \emph{literally} say ``your
name'' and the case where we actually ask what ``your name'' \emph{is}, we
use quotation marks in English. Similarly, quoting a symbol in
Lisp tells the interpreter to check \emph{literally} for \texttt{(car expr)}
to be the symbol \texttt{+} and not the procedure \texttt{+}.

Quotation is actually quite a complicated thing. Following the
principle of substituting equals for equals, consider:

\begin{verse}
``Chicago'' has seven letters.\\
Chicago is the biggest city in Illinois.\\
``The biggest city in Illinois'' has seven letters.\\
\end{verse}

The first two statements are true, and quotation marks are used
correctly in the first to show that we're talking about Chicago
the word and not Chicago the city. However, the third statement is
wrong entirely (although it is the result of changing equals for
equals), because the phrase ``The biggest city in Illinois'' does
not have seven letters.
That is, we cannot substitute equals for equals in referentially
opaque contexts.

Note that the \texttt{'} symbol breaks the neat pattern of Lisp where all
expressions are delimited by \texttt{()}. To resolve this, we introduce
the special form \texttt{(quote +)}, which does the exactly same thing as
\texttt{'+}.

Now defining the constructors:

\begin{minted}[]{racket}
(define (make-sum a1 a2)
  (list '+ a1 a2))

(define (make-product m1 m2)
  (list '* m1 m2))
\end{minted}


Finally, we must define the selectors:

\begin{minted}[]{racket}
(define a1 cadr)
(define a2 caddr)

(define m1 cadr)
(define m2 caddr)
\end{minted}


\texttt{cadr} is the \texttt{car} of the \texttt{cdr} and \texttt{caddr} is the \texttt{car} of the
\texttt{cdr} of the \texttt{cdr}. These are forms provided for convenience while
programming, since list processing a big part of Lisp.\footnote{LISP actually stands for LISt Processing.}

Let's try it out:

\begin{minted}[]{racket}






(deriv '(+ (* a (* x x)) (+ (* b x) c)) 'x)
(deriv '(+ (* a (* x x)) (+ (* b x) c)) 'a)
(deriv '(+ (* a (* x x)) (+ (* b x) c)) 'b)
(deriv '(+ (* a (* x x)) (+ (* b x) c)) 'c)
\end{minted}

\begin{verbatim}
(+ (+ (* a (+ (* x 1) (* 1 x))) (* 0 (* x x))) (+ (+ (* b 1) (* 0 x)) 0))
(+ (+ (* a (+ (* x 0) (* 0 x))) (* 1 (* x x))) (+ (+ (* b 0) (* 0 x)) 0))
(+ (+ (* a (+ (* x 0) (* 0 x))) (* 0 (* x x))) (+ (+ (* b 0) (* 1 x)) 0))
(+ (+ (* a (+ (* x 0) (* 0 x))) (* 0 (* x x))) (+ (+ (* b 0) (* 0 x)) 1))
\end{verbatim}


Note the recursive nature of \texttt{deriv}: the process creates results
with the same shape even when we differentiate with respect to
some other variable. This is because the recursion only ends when
an expression is decomposed to either \texttt{same-var?} or \texttt{constant?}.

\subsection{Simplification}
\label{sec:org04bdcbb}

However, these results are ugly, and we know why --- there's no
simplification. Technically, it's correct:

\begin{align*}
&a(1x+1x) + 0x^{2} + b + 0x + 0\\
=& 2ax + b
\end{align*}

Note that we've faced this same problem before with fractions, and
recall that the solution was to change the constructors so that
they'd simplify while creating the lists. Consider:

\begin{minted}[]{racket}
(define (make-sum a1 a2)
  (cond ((and (number? a1)
              (number? a2))
         (+ a1 a2))
        ((and (number? a1)
              (= a1 0))
         a2)
         ((and (number? a2)
              (= a2 0))
          a1)
         (else
          (list '+ a1 a2))))

(define (make-product m1 m2)
  (cond ((and (number? m1)
              (number? m2))
         (* m1 m2))
        ((and (number? m1)
              (= m1 0))
         0)
         ((and (number? m2)
              (= m2 0))
          0)
         ((and (number? m1)
               (= m1 1))
          m2)
         ((and (number? m2)
               (= m2 1))
          m1)
         (else
          (list '+ m1 m2))))
\end{minted}

Now trying \texttt{deriv}:


\begin{minted}[]{racket}






(deriv '(+ (* a (* x x)) (+ (* b x) c)) 'x)
(deriv '(+ (* a (* x x)) (+ (* b x) c)) 'a)
(deriv '(+ (* a (* x x)) (+ (* b x) c)) 'b)
(deriv '(+ (* a (* x x)) (+ (* b x) c)) 'c)
\end{minted}

\begin{verbatim}
(+ (+ a (+ x x)) b)
(* x x)
x
1
\end{verbatim}


Excellent, these are much better. Note, of course, that we could
simplify the first one further, but, in general, algebraic
simplification is a painful problem, since the definition of
simplest form varies with application. However, this is good
enough.

\section{On Abstract Syntax}
\label{sec:org4a7c095}

Note that the syntax we used was abstract in the sense that it had
its own rules and grammar. However, since it followed Lisp's syntax
closely, we needed quotation to allow full expression.

This is a powerful paradigm: not only can we use meta-linguistic
abstraction to create languages embedded within Lisp, but we can
also use Lisp to interpret any syntax. We'll see more of this in
the future.

\chapter{Lecture 4A: Pattern Matching and Rule-Based Substitution}
\label{sec:org8d5cced}

It's a funny technique we used last time, converting the rules of
differentiation to Lisp. In fact, if we wanted to explain (say) the
rules of algebra to the computer, we'd have to again create a
similar program which converts the rules of algebra to Lisp.

See that there's a higher-order idea here, of explaining rules to
Lisp and having the rules applied to an input expression to
``simplify'' it. Our style of writing a rule-based substitution
program is:

Rules \(\rightarrow\) conditional \(\rightarrow\) dispatch

That is, we try the rules on the given expression. If there's a
match, we ``dispatch'' the result to substitute. Now, in general, the
application of a rule is:

\begin{itemize}
\item Compare LHS of rule to input expression.
\item If match, RHS with substituted values is replacement.
\end{itemize}

Or, diagrammatically:

\[\begin{tikzcd} \mathrm{Pattern} \arrow{r}{\mathrm{Rule}}
  \arrow[swap]{d}{\mathrm{Matched}} & \mathrm{Skeleton}
  \arrow{d}{\mathrm{Instantiation}} \\ \mathrm{Expression_{Src}}
  \arrow[mapsto]{r} & \mathrm{Expression_{Target}} \end{tikzcd} \]

Let us now build a simple language to express these rules, which can
then be pattern matched, skeletons created, then instantiated.

\section{Rule Language}
\label{sec:orgc236be0}

Here's a sample bit of what we want the rule language to look like:

\begin{minted}[]{racket}
(define deriv-rules
  '(
    ((dd (?c c) (? v)) 0)
    ((dd (?v v) (? v)) 1)
    ((dd (?v u) (? v)) 0)

    ((dd (+ (? x1) (? x2)) (? v))
     (+ (dd (: x1) (: v))
        (dd (: x2) (: v))))

    ((dd (* (? x1) (? x2)) (? v))
     (+ (* (: x1) (dd (: x2) (: v)))
        (* (: x2) (dd (: x1) (: v)))))
    ; ...
    ))
\end{minted}

It is worth explaining what this syntax means exactly, because
eventually, we want to parse it.

The rules are a list of pairs. The \texttt{car} of each pair is the
pattern to match (rule LHS), and the \texttt{cdr} is the skeleton
substitution expression (rule RHS).

\subsection{Pattern Matching}
\label{sec:orgfae9fc4}

The idea of the LHS language is to provide a framework where
certain constructs can be matched and possibly named. These names
will then be passed to the skeleton instantiator.\footnote{We use ``initiate'' and ``substitute'' interchangeably to mean
swapping out expressions in the skeleton provided by the RHS of the
rules.}

\begin{center}
\begin{tabular}{ll}
\toprule
Syntax & Meaning\\
\midrule
\texttt{foo} & Matches itself literally.\\
\texttt{(f a b)} & Matches every 3-list whose \texttt{car} is \texttt{f}, \texttt{cadr} is \texttt{a}, and \texttt{caddr} is \texttt{b}.\\
\texttt{(? x)} & Matches any expression, and calls it \texttt{x}.\\
\texttt{(?c x)} & Matches an expression which is a constant, and calls it \texttt{x}.\\
\texttt{(?v x)} & Matches an expression which is a variable, and calls it \texttt{x}.\\
\bottomrule
\end{tabular}
\end{center}


\subsection{Skeleton and Instantiation}
\label{sec:orgfedf347}
The RHS language provides a skeleton wherein values provided by
the LHS language can be substituted.

\begin{center}
\begin{tabular}{ll}
\toprule
Syntax & Meaning\\
\midrule
\texttt{foo} & Instantiates \texttt{foo}.\\
\texttt{(f a b)} & Instantiates each element of the list and returns a list.\\
\texttt{(: x)} & Instantiate the value of \texttt{x} provided by the pattern matcher.\\
\bottomrule
\end{tabular}
\end{center}

\section{Sample Usage}
\label{sec:org0964e2c}

We expect to use this program by calling a procedure called
\texttt{simplifier}, to which we provide the list of rules. The procedure
should return another procedure, which is able to apply the rules
to a given input expression. Or, in Lisp:

\begin{minted}[]{racket}
(define dsimp
  (simplifier deriv-rules))

(dsimp '(dd (+ x y) x))
\end{minted}

\begin{verbatim}
(+ 1 0)
\end{verbatim}

\section{Implementation}
\label{sec:org6398556}
\end{document}